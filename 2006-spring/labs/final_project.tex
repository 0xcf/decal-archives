\documentclass[10pt]{article}
\usepackage{html}
\begin{document}
\title{System Administration\\
Final Project}

\maketitle
\section{Project Description}

The final project will consist of two parts: setup of a LAMP server
and simulation of a real-world server. In the first part, students
will install all the components of LAMP and secure the operating system
and daemons. The second part will be divided into two activities:
creation of four untrusted user accounts on the system and attempts
at being malicious users on another system.


\section{Project Specification}

\subsection{Setup of a LAMP Server }

The following services and programs should be installed and working
on the server: 

\begin{itemize}
\item SSH on port 1\texttt{XX}22
\item Apache with suExec on port 1\texttt{XX}80
\item PHP running as CGI under suExec
\item MySQL
\end{itemize}
where \texttt{XX} is your group number. You should also install the
following programs on the server

\begin{itemize}
\item \texttt{nmap} - a port scanning program. This program should only
be accessible to administrators.
\item \texttt{links} - a text-mode web browser.
\item \texttt{mutt} - a text-mode mail client.
\end{itemize}
Quotas and access control lists should be working, and the tools for
working with access control lists should be available to users on
the system.

\subsection{Server Simulation}


\subsubsection{Preparing Your Server}

You need to create four user accounts: two with complete shell access
and two that can only connect via \texttt{scp} and \texttt{sftp}.
Each account should also be provided with complete access to a MySQL
database. Please mail the login information and MySQL information
for these accounts to the \texttt{dima}+\texttt{decal}@\texttt{ocf.berkeley.edu}.

You should assume that all user accounts will be used by malicious
users. In other words, you should assume that a user account poses
a threat to your server and other users. Therefore, you should set
appropriate restrictions on each account. However, the accounts should 
be able to host their own website and execute PHP scripts that make
use of their MySQL database. 

You will be responsible for auditing user accounts and ensuring that
they have not set insecure permissions on their files. Scripts that check for insecure permissions are highly recommended. You should set the
scripts to execute on a regular basis, and have the output of these
scripts sent to your email addresses. You may even elect to have the
scripts automatically adjust file permissions.

If users manage to break their account in some way, it will be your
responsibility to help them fix their account, within the guidelines
of the project. 


\subsubsection{Testing Another Server}

You will be provided with four user accounts on another group's server.
One of these accounts will have complete access to the system, and
the other account will only have \texttt{scp} and \texttt{sftp} access
to the system. Your task will be to act as malicious users and test
the security of the server. You should attempt to gain unauthorized
access to files and directories, exceed any quotas placed upon your
accounts, execute restricted commands, and any other attack within
the project guidelines.

You will also be testing the server to ensure that you can use your
web space to run PHP scripts that interact with your MySQL database.
See the list of suggested scripts to setup below. Please note
that if you are using a script of your own choosing, the administrators
of the server will not be responsible for any security holes that
develop as a result of the script, beyond those involved with file
permissions. 


\section{Project Guidelines}

\begin{itemize}
\item All submission text should be in your own words. For example, do not
just copy and paste the explanation for a command from its man page,
and do not just copy and paste the package description from \texttt{apt-cache}
when describing a package. 
\item Attacks against remote systems, other than the servers to which you
have been granted access, should not be used and are in violation
of campus network policy.
\item Attacks that consume all the processing power of a server should not
be used and will not be considered valid.
\end{itemize}

\section{Project Hints and Help}

\begin{itemize}
\item If you need help or have any questions, please send an email with
your group number to \texttt{sysadmin-decal@ocf.berkeley.edu}.
\item When choosing a package to install, it is usually best to choose the
package with the simplest name. For example, \texttt{apt-cache} presents
you with the choice between installing \texttt{foo-server} and \texttt{foo-server10},
\texttt{foo-server} will probably be the best package to install.
\item If you wish to completely remove a package, recall that you may pass
the \texttt{-{}-purge} parameter to apt-get. For example, to completely
remove package \texttt{bar},
\end{itemize}
\subsection{Suggested Internet Scripts}

The following are PHP scripts that make use of a MySQL database. Please note that the installation documentation for some of these scripts specifies insecure actions that you should not perform blindly; using what you know, figure out a more secure method.

\begin{itemize}
\item gallery: \htmladdnormallink{gallery.sf.net}{http://gallery.sf.net}
\item wordpress: \htmladdnormallink{www.wordpress.org}{http://www.wordpress.org}
\item drupal: \htmladdnormallink{www.drupal.org}{http://www.drupal.org}
\end{itemize}

\section{Submission Guidelines }
Please submit a listing of all important commands you used to setup your server, an short explanation for each command, and whether root access was needed to execute the command. If necessary, provide the context in which you used the command (ex., the directory you were in when you executed the command) or the output of a command.  With regards to package installation, you should also provide a description of why it was necessary to install the package and, if applicable, why you chose it instead of other similar packages. 
\section{Project Grading}
The grading system for the project will be finalized shortly. For now, assume that both parts of the project are equally weighted.
\end{document}
