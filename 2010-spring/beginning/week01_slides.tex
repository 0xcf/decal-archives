\documentclass[14pt]{beamer}

\usetheme{Warsaw}
\usecolortheme{lily}
\usepackage{graphicx}

\title{Beginning and Intermediate System Administration DeCal}
\author{Week 1}
\date{February 1, 2010}

\begin{document}

\begin{frame}
\maketitle
\end{frame}

\begin{frame}
	\frametitle{Introduction}
	\begin{itemize}
		\item CS98/198-15: Beginning and Intermediate System Administration DeCal
		\begin{itemize}
		  \item Sponsored by the Open Computing Facility 
		  \item Offered since way back when (2001-ish)
		\end{itemize} 
	\end{itemize}
\end{frame}

\begin{frame}
	\frametitle{Who are We?}
	\begin{itemize}
		\item Four facilitators
		\begin{itemize}
			\item Sanjay (\texttt{sanjayk+decal@ocf.berkeley.edu})
			\item Alan (\texttt{alanw+decal@ocf.berkeley.edu})
      \item Michael (\texttt{mgasidlo+decal@ocf.berkeley.edu})
			\item Jordan (\texttt{jordan+decal@ocf.berkeley.edu})
		\end{itemize}
		\item Office Hours: Listed on site below, also by appointment
		\begin{itemize}
			\item \texttt{www.ocf.berkeley.edu/staff\_hours}
			\item Direct the administrative questions to us
			\item Don't hesitate to contact us about questions and concerns!
		\end{itemize}
	\end{itemize}
\end{frame}

\begin{frame}
  \includegraphics[scale=0.5]{desktop-computer.jpg}
\end{frame}

\begin{frame}
  \includegraphics[scale=0.25]{dedicatedservers.jpg}
  \includegraphics[scale=0.35]{pcservereatindata.jpg}
\end{frame}

\begin{frame}
  \includegraphics[scale=0.7]{netcableunplugged.jpg}
\end{frame}

\begin{frame}
  \includegraphics[scale=0.75]{linuxMotPoster.jpg}
\end{frame}

\begin{frame}
  \includegraphics[scale=0.35]{tubes.jpg}
\end{frame}

\begin{frame}
	\frametitle{What and Why?}
	\begin{itemize}
		\item System Administration
		\begin{itemize}
			\item Not necessarily how to use a system
			\item GUIs, how to get your modem working, etc... not covered in class. But the skills sets are!
			\item The "magic" behind how computer labs (Like \texttt{inst}) work.
		\end{itemize}
		\item Aspiring sysadmin
		\item Marketable skills (maybe)
		\item Manage your own resources
	\end{itemize}
\end{frame}

\begin{frame}
	\frametitle{What and Why?}
	\begin{itemize}
		\item Further understanding of ``how stuff works''
		\begin{itemize}
			\item Internet servers
			\item Security, encryption, and why it matters
			\item Communication on the network 
		\end{itemize}
		\item Two sections offered this semester: beginning and intermediate
		\begin{itemize}
		\item For beginners, black box implementation; we'll worry about the initial setup.
			\item For the intermediate, dig a little deeper.
		  \item Check the DeCal website for more information
		\end{itemize}
	\end{itemize}
\end{frame}

\begin{frame}
	\frametitle{Prerequisites}
	\begin{itemize}
		\item None!
		\begin{itemize}
			\item But you may find things a bit easier if you've got some groundwork already. (Interaction with \texttt{inst} accounts)
			\item It would help to learn a text editor; the sooner the better. 
			\item A few choices are \texttt{nano, vim, emacs}.
			\item \texttt{vimtutor} is an excellent guide to learn \texttt{vim}.
		\end{itemize}
	\end{itemize}
\end{frame}

\begin{frame}
	\frametitle{Expectations}
	\begin{itemize}
	  \item Check the website for the latest updates!
		\item Come to class
		\begin{itemize}
			\item Don't sleep (Try not to)
		\end{itemize}
		\item Do the reading
		\begin{itemize}
		  \item Material builds on each other
		  \item Helps to read the material and ask about topics while it's fresh in lecture
		  \item Print out the notes (\texttt{inst} accounts) and mark them up
		\end{itemize}
		\item Do the work
		\begin{itemize}
			\item CS98/198, 2 units = ~6 hours of work (at least!)
			\item \url{http://slc.berkeley.edu/ucftr/docs/unit_value.pdf}
			\item Participation and effort in lab/homework. 
		\end{itemize}
	\end{itemize}
\end{frame}

\begin{frame}
  \frametitle{Expectations}
  \framesubtitle{Continued}
  \begin{itemize}
  	\item Work with your classmates and peers
		\item Ask questions in class, through email, in office hours\ldots
		\item Seek help from all the resources available to you
		\begin{itemize}
		  \item Peers
		  \item Facilitators
		  \item Internet
		\end{itemize}
	\end{itemize}
\end{frame}

\begin{frame}
	\frametitle{Logistics and Policies}
	\framesubtitle{Course Format}
	\begin{itemize}
%		\item 2 hours \emph{reserved} each week
		\item Lecture: 0.5-1 hour
		\item Lab: Rest of the time
    \item Beginner will start here (310 Soda) for lecture
    \item Intermediate go straight to lab (271 Soda)	
		\item Homework: assigned weekly (most of the time)
		\item Grading
		\begin{itemize}
			\item P/NP basis, final project 50\%, labs 30\%, homework 20\%
			\item Submissions policy
			\item \textbf{Final project is mandatory!}
		\end{itemize}
	\end{itemize}
\end{frame}

\begin{frame}
	\frametitle{Topics}
	\begin{itemize}
		\item History of UNIX, GNU/Linux
		\item Command line interface (CLI)
		\item The Internet
		\item Server Daemons (web, databases, etc)
		\item The LAMP model
		\item Multi-user environments
		\item Security
		\item \emph{Special topics?}
		\begin{itemize}
		  \item \textbf{Your} input is needed!
		\end{itemize}
	\end{itemize}
\end{frame}

\begin{frame}
	\frametitle{Homework \#1}
	\begin{itemize}
		\item Create an OCF account
		\begin{itemize}
			\item Lab is located in Eshleman Hall, floor G
			\item \url{http://www.ocf.berkeley.edu/lab/labdirectionsocf.jpg}
			\item Do this before Thursday to ensure your account is created this week.
			\item \url{http://www.ocf.berkeley.edu}
		\end{itemize}
		\item Fill out the course survey (online)
		\begin{itemize}
		  \item \url{http://www.surveymonkey.com/s/KMQFF9B}
		\end{itemize}
	\end{itemize}
\end{frame}

\begin{frame}
	Questions?
\end{frame}

\end{document}
