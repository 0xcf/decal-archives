%% LyX 1.3 created this file.  For more info, see http://www.lyx.org/.
%% Do not edit unless you really know what you are doing.
\documentclass[english]{article}
\usepackage[T1]{fontenc}
\usepackage[latin1]{inputenc}

\makeatletter
%%%%%%%%%%%%%%%%%%%%%%%%%%%%%% Textclass specific LaTeX commands.
 \newenvironment{lyxcode}
   {\begin{list}{}{
     \setlength{\rightmargin}{\leftmargin}
     \setlength{\listparindent}{0pt}% needed for AMS classes
     \raggedright
     \setlength{\itemsep}{0pt}
     \setlength{\parsep}{0pt}
     \normalfont\ttfamily}%
    \item[]}
   {\end{list}}

\usepackage{babel}
\makeatother
\begin{document}

\title{System Administration for the Web:\\
Week 7 Lab}


\date{October 24, 2005}

\maketitle

\section{Notes}

This laboratory focuses entirely upon programming in Perl. We understand
that many of you have had no programming experience at all, so we've
tried to make this laboratory as simple as possible. However, don't
feel too bad if you don't completely understand what you're doing
-- we aren't expecting you to learn how to program in just 40 minutes.
\emph{You should be able to complete these exercises, though}. We're
planning on making one part of the project a programming exercise,
so it's in your best interests to at least attempt to complete this
assignment and ask questions if you run into trouble.


\section{Running Perl Programs}

To a computer, Perl programs are just ordinary text files. They only
know to run a Perl program if you give your Perl program the appropriate
permissions. In this case, the minimum permissions are read and execute.

If you've correctly specified the \emph{shebang} line and set the
correct permissions, executing your Perl program is as simple as typing
something like the following:

\begin{lyxcode}
./my\_perl\_program
\end{lyxcode}
in the directory that contains your Perl program. In the off-chance
that you've accidentally created an infinite loop in your program,
please recall that you can press \texttt{Ctrl-C} to terminate the
execution of a program.


\section{Lab for Week 7}


\subsection{Modifying A Simple Perl Program}

The first half of this laboratory will be modifying the guessing number
game presented during lecture and in the lecture notes. I've conveniently
placed a copy of this program in the \texttt{/home/cc/cs198/fa05/class/cs198-ec/labs/lab7}
directory.

\begin{description}
\item [{[}1{]}]Copy the \texttt{guess\_the\_number.pl} file to your home
directory.
\item [{[}2{]}]Modify the program to ask the user for the maximum random
number.
\item [{[}3{]}]Modify the program to count the number of guesses a person
uses and to print this number when the user guesses the correct number.
\end{description}

\subsection{Writing Your Own Programs}

\begin{description}
\item [{[}1{]}]Write a program that prompts the user for and reads two
numbers (on separate lines of input) and prints out the product of
the two numbers multiplied together.
\item [{[}2{]}]Write a program that will ask the user for a given name
and report the corresponding family name. Use a hash of given and
family names in your program.
\end{description}

\end{document}
